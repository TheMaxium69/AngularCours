\documentclass[10pt]{beamer}

% Package
\usepackage{graphicx}

% Theme
\usetheme{Warsaw}
\usecolortheme{beaver}

% Main & Info
\title[Angular]
{Cour Angular}
\subtitle{FrameWork JS}
\author[Maxime Tournier]
{Maxime Tournier}
\date[21/08/2023]

% Start
\begin{document}

	\frame{\titlepage}
%	\begin{frame}
%		\frametitle{Demonstrating}
%
%		In this slide, some important text will be
%		\alert{highlighted} because it's important.
%		Please, don't abuse it.
%
%
%		\includegraphics[width=1cm]{C:/Users/mxmto/Pictures/63310746.png}
%
%		\begin{block}{Remark}
%			Sample text
%		\end{block}
%
%	\end{frame}

	\begin{frame}
		\frametitle{Sommaire}

		1. {Présentation}


	\end{frame}

	\begin{frame}
		\frametitle{Présentation}
	
		Angular est un FrameWork

		\begin{block}{Traduction}
			Framework = Cadre de travail
		\end{block}

		En tant que développeur on fait souvent la meme chose \newline \newline
		Exemple: \newline Valider les formulaires | Gerez la navigation | traiter les erreurs
		\newline \newline
		Et pour regler ce problème au lieu d'allez recupere les fonction d'autre projet on à créer les framework
		\newline \newline
		Avantage : Tout le monde travaille sur les même fonction
		
	\end{frame}

	\begin{frame}
		\frametitle{Présentation}

		Pour comprendre le fonctionnement d'angular
		\newline \newline
		Il faut comprendre le web aujourd'hui
		\newline \newline
		Site Web / Application Web
		\newline \newline
		Il s'agit bien de deux chose différente

	\end{frame}

	\begin{frame}
		\frametitle{Présentation}

		Avant ça comment ça marche :
		\newline \newline
		- Fichier serveur (php, java) (ici qu'on fait des requetes sql)\newline
		- Fichier client (CSS, HTML, JS)

		(Shema)

	\end{frame}

	\begin{frame}
		\frametitle{Présentation}

		Et donc un site web fonctionne comme ça :
		\newline \newline

		(Shema)

	\end{frame}

	\begin{frame}
		\frametitle{Présentation}

		Et une application web marche comme ça:
		\newline \newline

		(Shema)

	\end{frame}

\end{document}